%% Согласно ГОСТ Р 7.0.11-2011:
%% 5.3.3 В заключении диссертации излагают итоги выполненного исследования, рекомендации, перспективы дальнейшей разработки темы.
%% 9.2.3 В заключении автореферата диссертации излагают итоги данного исследования, рекомендации и перспективы дальнейшей разработки темы.
применение машинного обучения приносит свои плоды, однако внедрение затруднено тем, что клиенты не готовы доверять автоматизированным системам такого уровня абстракций. Трактовка работы моделей на базе машинного обучения и визуализация результата является необходимой частью успешного проекта. При этом желательно, чтобы трактовка происходила в реальном времени, с привлечением экспертов рабочей группы с самого начала разработки системы.
Опыт применения методов машинного обучения к задачам управления инфраструктурой российских железных дорог на Московской железной дороге  позволяет  ожидать  положительные  практические результаты при дальнейшем развитии и тиражировании достигнутых совместной рабочей группой специалистов.
По  результатам  доклада  генерального  директора  компании  ООО «Телум»  Павла  Бойко, Ученый  совет  ОАО «РЖД»  признал  перспективным применение современных методов машинного обучения к задачам управления инфраструктурой российских железных дорог и порекомендовал продолжить развитие, разработку и внедрение си-стемы в составе более широкой совместной груп-пы, с вовлечением отраслевых институтов и университетов (в т.ч. АО«ВНИИЖТ»). Представленные наработки в области интеллектуального анализа данных также рекомендовано апробировать в других областях железнодорожного транспорта.
\begin{enumerate}
  \item \todo{На основе анализа \ldots}
  \item \todo{Численные исследования показали, что \ldots}
  \item \todo{Математическое моделирование показало \ldots}
  \item \todo{Для выполнения поставленных задач был создан \ldots}
\end{enumerate}
