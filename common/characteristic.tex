
{\actuality} \todo{Обзор, введение в тему, обозначение места данной работы в
мировых исследованиях и~т.\:п.}
В данной работе предлагается рассмотреть несколько примеров использования машинного обучения в задачах управления инфраструктурой. Далее предлагается обобщить опыт и предложить метод построения и трактовки моделей, полученных при помощи алгоритмов машинного обучения в условиях современной инфраструктуры.
Исследованию применения методов машинного обучения к задачам управления инфраструктурой посвящено значительное количество работ, среди которых следует особо отметить работы российских и зарубежных ученых: Кулешов А. П., Лаптев В. Н., Сидельников О. В., Шарай В. А. Бериков В., Лбов Г., Прищепа М. В.,Нестеров В. В., Ефанов Д. В, Herbst J., Magnusson J., Kvernik T., Hung S.-Y., Yen D. C., Wang H.-Y., Halpern S. W., Liu S., Virtamo J., Hu H., Luo J., Chen H.-H., Choi W., Andrews J. G., Leith D., Clifford P. Однако они работают в традиционном подходе, обозначенном выше; они не рассчитывают эффективность всей системы, так же нет выработки решающего правила, которое после трактуется экспертами, что и определило направление исследований, выполненных в диссертации.
В целом, существующий подход можно назвать традиционным, основанным на построении аналитических моделей, и отличающийся следующими особенностями:
\begin{itemize}
\item  сложная система разбивается на несколько независимых уровней
\item  определяются показатели эффективности каждого уровня
\item  производится построение, валидация и верификация <аналитических> моделей каждого уровня
\item  показатели каждого уровня оптимизируются в модельных сценариях
\item  производится подстройка модели
\end{itemize}
В то время как предлагаемый инновационный подход основан на машинном обучении и генерации признаков и характеризуется следующим:
\begin{itemize}
\item  сложная система данных воспринимается как "серый ящик"
\item  определяются показатели эффективности системы в целом
\item  определяется конечное множество состояний системы
\item  из базовых признаков состояния системы производится порождение составных признаков
\item  вырабатывается решающее правило
\item  решающее правило и модель принятия решения валидируется экспертами (трактовка результатов)
\end{itemize}
В решении поставленных в исследовании задач одним из средств станет концепция АСУ ТП или, говоря иначе, Internet of things.
\todo{Структура экономик развитых стран (аналогично предыдущим периодам – индустриализации и компьютеризации). Этот период обусловлен бурным внедрением новых технологий в области мобильной радиосвязи, передачи данных и развития сети Интернет, идущим с начала 2000-х. Этот период можно условно назвать цифровизацией экономики и для его начала должна быть достигнута определенная критическая масса новых технологий, внедренных в повседневную жизнь, бизнес и промышленность.}
\todo{Одним из наиболее эффективных инструментов в достижении нового уровня цифровизации может стать «Интернет вещей» (Internet of Things, IoT), а также развитие беспроводных технологий передачи данных, лежащих в основе транспортной инфраструктуры интернета вещей. В мире продолжает расти количество «подключенных» устройств (по оценкам отраслевых аналитиков, к 2020 году их будет от 20 до 30 млрд единиц), и вместе с ним увеличивается число примеров применения интернета вещей в экономике: энергетике, промышленности, жилищно-коммунальном хозяйстве, сельском хозяйстве, транспорте, здравоохранении и др.}
\todo{У интернета вещей есть важные преимущества перед другими прорывными технологиями. Во-первых, IoT-технологии могут широко применяться как для обслуживания конечных потребителей, так и в бизнесе в целом в самых широких областях применения. С другой стороны, для начала использования IoT уже есть в той или иной степени готовая инфраструктура – мобильные и фиксированные сети, а дальнейшее внедрение (сенсоры, приложения, платформы) достаточно дешево. Также, технологии IoT окажут быстрое мультипликативное воздействие на отрасли экономики за счет повышения производительности труда и сокращения затрат, а также за счет создания новых источников дохода для компаний.}



% разнородная по всем
% алгоритмов полно
% основные задачи



% , можно использовать ссылки на другие
% работы~\cite{Gosele1999161} (если их нет, то в автореферате
% автоматически пропадёт раздел <<Список литературы>>). Внимание! Ссылки
% на другие работы в разделе общей характеристики работы можно
% использовать только при использовании \verb!biblatex! (из-за технических
% ограничений \verb!bibtex8!. Это связано с тем, что одна и та же
% характеристика используются и в тексте диссертации, и в
% автореферате. В последнем, согласно ГОСТ, должен присутствовать список
% работ автора по теме диссертации, а \verb!bibtex8! не умеет выводить в одном
% файле два списка литературы).

% {\progress}
% Этот раздел должен быть отдельным структурным элементом по
% ГОСТ, но он, как правило, включается в описание актуальности
% темы. Нужен он отдельным структурынм элемементом или нет ---
% смотрите другие диссертации вашего совета, скорее всего не нужен.

{\aim} данной работы является является создание новых методов (т.е. предсказательных математических моделей, алгоритмов и комплексов программ) совмещения телекоммуникационных, измерительных и управляющих систем, новых методов исследования, моделирования и проектирования эргатических систем, пригодных для интеграции с современными индустриальными программно-аппаратными комплексами с применением машинного обучения и их применение для построения систем управления.


Для~достижения поставленной цели необходимо было решить следующие {\tasks}:
\begin{enumerate}
  \item Обзор и анализ существующих алгоритмов управления инфраструктурой на основе оперативных массивов данных
  \item Разработка и исследование методов и алгоритмов управления радио-ресурсами в телекоммуникационных сетях четвертого поколения (4G)
  \item Разработка и исследование методов и алгоритмов управления приоритетом обработки инцидентов на основе оперативных данных устройств железнодорожной автоматики и телемеханики
  \item Экспериментальное исследование характеристик разработанных моделей, обобщение на более широкий класс задач

\end{enumerate}

{\novelty}
В диссертации впервые:
\begin{enumerate}
\item Разработана новая аналитическая модель, позволяющая оптимизировать поиск эффективного режима распределения ресурсов телекоммуникационной инфраструктуры сетей пятого поколения при помощи Q-обучения без учителя, которая также позволяет моделировать системный и физический уровни перспективной телекоммуникационной инфраструктуры.
\item Разработан оригинальный метод построения предсказательных моделей эргатических систем, пригодных для интеграции с современными индустриальными программно-аппаратными комплексами на базе машинного обучения. Повышенная эффективность данных моделей используется для анализа и разработки современных комплексов программ по управлению критической инфраструктурой, также с применением методов машинного обучения как с учителем, так и без него.
\item Предложен новый алгоритм управления радио-ресурсами в телекоммуникационных сетях пятого поколения, использующийся в целях снижения общего уровня интерференции и максимизации суммарной пропускной способности в сети базовых станций.
\end{enumerate}
Полученные данные объединены в ряд внедренных методических рекомендаций, что подтверждено соответствующими актами о внедрении.
Все результаты являются новыми и получены в соответствующих работах впервые.

{\influence} \ldots
\todo{Результаты настоящего исследования могут быть применены для автоматизации более эффективного распределения ресурсов телекоммуникационной инфраструктуры сетей пятого поколения, путем  имитационного моделирования системного и физического уровней перспективной телекоммуникационной инфраструктуры на установленную потребность для обеспечения ориентированного на требования качества обслуживания коммуникационного процесса. Построенные системы управления в настоящее время частично введены в опытную эксплуатацию компанией ООО «Техкомпания Хуавэй» и ДИТ города Москвы,  что подтверждено соответствующими документами. Кроме того, исследование по управлению приоритетом обработки инцидентов, выполненное в данной работе, позволяет использовать полученные результаты для контроля обработки оперативных данных устройств железнодорожной автоматики и телемеханики в ОАО “РЖД”, что подтверждено актом о внедрении.}.


{\methods} Для решения поставленных задач используются методы прикладной математики, теории вероятностей, теории массового обслуживания, вычислительной математики и методы математического и имитационного моделирования, а также искусственные нейронные сети, байессовские ядерные методы, методы на основе лесов решающих деревьев, методы машинного обучения  с учителем и без учителя, q-обучение, компьютерные методы обработки информации и моделирования, методы теории графов, методы разработки приложений на языках программирования python, jsx, react и языках семейства web 2.0. Для решения ряда пратических задач использовались  специальные методы из области АСУ ТП, обработки естественного языка, теории сетей связи.


{\defpositions}
\begin{enumerate}
  \item Комплекс программ для имитационного моделирования системного и физического уровней перспективной телекоммуникационной инфраструктуры.
  \item Методы построения предсказательных моделей эргатических систем, пригодных для интеграции с индустриальными программно-аппаратными комплексами на базе машинного обучения.
  \item Децентрализованный алгоритм управления радио-ресурсами в телекоммуникационных сетях пятого поколения, использующийся в целях снижения общего уровня интерференции и максимизации суммарной пропускной способности в сети базовых станций.
  \item Адаптированный под требования по качеству обслуживания пользователей алгоритм оптимизации для решения задачи поиска эффективного режима распределения ресурсов телекоммуникационной инфраструктуры сетей пятого поколения. В основе алгоритма лежит метод Q-обучения без учителя.
  \item Комплекс программ для управления приоритетом обработки инцидентов на основе оперативных данных устройств железнодорожной автоматики и телемеханики с использованием методов машинного обучения с учителем.
  \item Методические подходы к повышению эффективности анализа и разработки современных комплексов программ для управления критической инфраструктурой с применением методов машинного обучения с учителем и без учителя.
\end{enumerate}
% В папке Documents можно ознакомиться в решением совета из Томского ГУ
% в файле \verb+Def_positions.pdf+, где обоснованно даются рекомендации
% по формулировкам защищаемых положений.

{\reliability} полученных результатов обеспечивается с помощью валидации проведенных вычислительных экспериментов. Иными словами, их проведение с соответствующими параметрами и в соответствующих сценариях позволяет воспроизвести существующие решения аналогичных задач в реальном масштабе и времени.



% Внедрение

Программная реализация предложенных алгоритмов и методов внедрены \todo{~\cite{act-rzd, act-huawei, act-heisen, act-common, act-aiwins, act-ditmsk}} и используются на практике, что подтверждено соответствующими документами. В частности, программная реализация системы автоматического ранжирования инцидентов внедрена в компании ОАО «РЖД», а разработанные методические подходы используются в компаниях АНООВО «Сколковский институт науки и технологий», ООО «Техкомпания Хуавэй», ДИТ города Москвы, ООО «Хайзен», ООО «КоммОН».

{\probation}
Основные результаты работы докладывались~на:
\begin{itemize}
\item 35-я конференция-школа молодых ученых и специалистов «Информационные технологии и системы – 2011», 1 – 6 сентября, 2011 года, г. Геленджик, Россия;
\item Международный симпозиум по беспроводному доступу «WiFlex», 4 - 6 сентября,  2013 года, г. Калининград, Россия;
\item Международная конференция «IEEE GLOBECOM 2015», 6 - 10 декабря, 2015 года, г. Сан-Диего, Калифония, США;
\item Международная конференция «International Conference on Big Data and its Applications», 16 - 17 сентября 2016, г. Москва, Россия;
\item 3 международная конференция «Engineering and Telecommunication Technologies 2016», 27 - 27 ноября, 2016 года, г. Долгопрудный, Московская область, Россия;
\item Научный семинар «Структурные Модели и Глубинное Обучение», ИППИ РАН, 6 декабря, 2016 года, г. Москва, Россия;
\item 4 международная конференция «Engineering and Telecommunication Technologies 2017», 29 - 30 ноября, 2017 года, г. Долгопрудный, Московская область, Россия;
\item 16-ая Международная конференция «IEEE International Conference On Machine Learning And Applications», 18 - 21 декабря, 2017 года, г. Канкун, Мексика.
\item «Winter Femto School by CTTC, ICT IP BeFemto (FP7) and ICT STREP Freedom», 6 - 10 февраля, 2012 года, г. Барселона, Каталония;
\item Лекция в Shukhov Lab «Internet is dead! Heil the Internet of Things», 14 мая, 2018 года, г. Москва, Россия;
\item Открытие «Точки кипения Челябинск» на базе Челябинского ИТ-парка, 26-27 февраля, 2018 года, г. Челябинск, Россия;
\item Научный семинар кружкового движения «Корпус», 28 февраля, 2018 года, г. Челябинск, Россия;
\item 5 международная конференция «Engineering and Telecommunication Technologies 2018», 15-16 ноября, 2018 года, г. Долгопрудный, Московская область, Россия;
\item Международная конференция компании Huawei совместно с ИППИ РАН «4th International Professor’s Day», 30 ноября-1 декабря, 2017 года, г. Москва, Россия;
\item Выступление на «Хакатон 2025», 27-28 января, 2018 года, г. Москва, Россия;
\item Выступление на «Четвертый Агрохакатон в сфере агродизайна и сити-фермерства», 27-29 апреля, 2018 года, г. Москва, Россия;
\item Доклад перед Ученым Советом ОАО «РЖД» (доклад выполнил генеральный директор компании ООО «Телум» Павел Бойко), 26 декабря, 2016 года, г. Москва, Россия.

\end{itemize}

{\contribution} Научные результаты, представленные в диссертационной работе, получены соискателем самостоятельно и при непосредственном участии.

%\publications\ Основные результаты по теме диссертации изложены в ХХ печатных изданиях~\cite{Sokolov,Gaidaenko,Lermontov,Management},
%Х из которых изданы в журналах, рекомендованных ВАК~\cite{Sokolov,Gaidaenko},
%ХХ --- в тезисах докладов~\cite{Lermontov,Management}.

\ifnumequal{\value{bibliosel}}{0}{% Встроенная реализация с загрузкой файла через движок bibtex8
    \publications\ Основные результаты по теме диссертации изложены в XX печатных изданиях,
    X из которых изданы в журналах, рекомендованных ВАК,
    X "--- в тезисах докладов.%
}{% Реализация пакетом biblatex через движок biber
%Сделана отдельная секция, чтобы не отображались в списке цитированных материалов
    \begin{refsection}%
        \printbibliography[heading=countauthornotvak, env=countauthornotvak, keyword=biblioauthornotvak, section=1]%
        \printbibliography[heading=countauthorvak, env=countauthorvak, keyword=biblioauthorvak, section=1]%
        \printbibliography[heading=countauthorconf, env=countauthorconf, keyword=biblioauthorconf, section=1]%
        \printbibliography[heading=countauthor, env=countauthor, keyword=biblioauthor, section=1]%
        \nocite{globecom,electosvyasen-2017, bulletin-rzd, itivs-2017, icmla-2017, ent, ent-2017, 5G}
        \nocite{ent, itivs-2017, electosvyasen-2017, ent-2017}
        \nocite{globecom, icmla-2017, bulletin-rzd, 5G}
        \publications\ Основные результаты по теме диссертации изложены в \arabic{citeauthor} печатных изданиях,
        \arabic{citeauthorvak} из которых изданы в журналах, рекомендованных ВАК,
        \arabic{citeauthorconf} "--- в тезисах докладов.
    \end{refsection}
}
