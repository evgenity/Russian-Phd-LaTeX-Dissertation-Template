
{\actuality} Обзор, введение в тему, обозначение места данной работы в
мировых исследованиях и~т.\:п.

Традиционный подход (построение аналитических моделей):
\begin{itemize}
\item  сложная система разбивается на несколько независимых уровней
\item  определяются показатели эффективности каждого уровня
\item  производится построение, валидация и верификация <аналитических> моделей каждого уровня
\item  показатели каждого уровня оптимизируются в модельных сценариях
\item  производится подстройка модели
\end{itemize}


Предлагаемый подход (на основе машинного обучения, генерация фич):
\begin{itemize}
\item  сложная система данных воспринимается как "серый ящик"
\item  определяются показатели эффективности системы в целом 
\item  определяется конечное множество состояний системы
\item  из базовых признаков состояния системы производится порождение составных признаков 
\item  вырабатывается решающее правило
\item  решающее правило и модель принятия решения валидируется экспертами (трактовка результатов)
\end{itemize}

В данной работе предлагается рассмотреть несколько примеров использования машинного обучения в задачах управления инфраструктурой. Далее предлагается обобщить опыт и предложить метод построения и трактовки моделей, полученных при помощи алгоритмов машинного обучения в условиях современной инфраструктуры.


% разнородная по всем
% алгоритмов полно
% основные задачи 



% , можно использовать ссылки на другие
% работы~\cite{Gosele1999161} (если их нет, то в автореферате
% автоматически пропадёт раздел <<Список литературы>>). Внимание! Ссылки
% на другие работы в разделе общей характеристики работы можно
% использовать только при использовании \verb!biblatex! (из-за технических
% ограничений \verb!bibtex8!. Это связано с тем, что одна и та же
% характеристика используются и в тексте диссертации, и в
% автореферате. В последнем, согласно ГОСТ, должен присутствовать список
% работ автора по теме диссертации, а \verb!bibtex8! не умеет выводить в одном
% файле два списка литературы).

% {\progress} 
% Этот раздел должен быть отдельным структурным элементом по
% ГОСТ, но он, как правило, включается в описание актуальности
% темы. Нужен он отдельным структурынм элемементом или нет ---
% смотрите другие диссертации вашего совета, скорее всего не нужен.

{\aim} данной работы является \ldots

Для~достижения поставленной цели необходимо было решить следующие {\tasks}:
\begin{enumerate}
  \item Обзор и анализ существующих алгоритмов управления инфраструктурой на основе оперативных массивов данных
  \item Разработка и исследование методов и алгоритмов управления радио-ресурсами в телекоммуникационных сетях четвертого поколения (4G)
  \item Разработка и исследование методов и алгоритмов управления приоритетом обработки инцидентов на основе оперативных данных устройств железнодорожной автоматики и телемеханики
  \item Экспериментальное исследование характеристик разработанных моделей, обобщение на более широкий класс задач

\end{enumerate}

{\novelty}
\begin{enumerate}
  \item Впервые \ldots
  \item Впервые \ldots
  \item Было выполнено оригинальное исследование \ldots
\end{enumerate}

{\influence} \ldots

{\methods} Для решения задач, поставленных в работе, были использованы основные положения системного анализа, теории информации, теории вероятностей; для проектирования программной системы – методы объектно-ориентированного проектирования и язык UML; для программной реализации алгоритмов и системы – методы структурного, объектно-ориентированного и параллельного программирования.

{\defpositions}
\begin{enumerate}
  \item Способ формирования модели планирования ресурсов телекоммуникационной инфраструктуры, пригодной для применения методов машинного обучения. Децентрализованный метод управления радио-ресурсами в телекоммуникационных сетях четвертого поколения в целях понижения общего уровня интерференции и максимизации суммарной пропускной способности в сети базовых станций.
  \item Алгоритм и режимы работы элементов телекоммуникационной инфраструктуры сетей четвертого поколения, учитывающие требования по  качеству обслуживания пользователей.
  \item Алгоритмов и архитектура управления приоритетом обработки инцидентов на основе оперативных данных устройств железнодорожной автоматики и телемеханики.
  \item Алгоритмов и архитектура управления приоритетом обработки инцидентов на основе оперативных данных устройств железнодорожной автоматики и телемеханики.
\end{enumerate}
% В папке Documents можно ознакомиться в решением совета из Томского ГУ
% в файле \verb+Def_positions.pdf+, где обоснованно даются рекомендации
% по формулировкам защищаемых положений. 

{\reliability} полученных результатов обеспечивается \ldots \ Результаты находятся в соответствии с результатами, полученными другими авторами.


% Внедрение

Программная реализация системы  внедрена... \todo{ссылка на отчет о внедрении}
Программная реализация системы автоматического ранжирования инцидентов внедрена \todo{ссылка на отчет о результатах подконтрольной эксплуатации}

{\probation}
Основные результаты работы докладывались~на:
\begin{itemize}
\item 35-я конференция-школа молодых ученых и специалистов «Информационные технологии и системы – 2011», 1 – 6 сентября, 2011 года, г. Геленджик, Россия;
\item Международная конференция «IEEE GLOBECOM 2015», 6 - 10 декабря, 2015 года, г. Сан-Диего, Калифония, США;
\item Международная конференция «International Conference on Big Data and its Applications», 16 - 17 сентября 2016, г. Москва, Россия;
\item 3 международная конференция «Engineering and Telecommunication Technologies 2016», 27 - 27 ноября, 2016 года, г. Долгопрудный, Московская область, Россия;
\item Научный семинар «Структурные Модели и Глубинное Обучение», ИППИ РАН, 6 декабря, 2016 года, г. Москва, Россия;
\item 4 международная конференция «Engineering and Telecommunication Technologies 2017», 29 - 30 ноября, 2017 года, г. Долгопрудный, Московская область, Россия;
\item 16-ая Международная конференция «IEEE International Conference On Machine Learning And Applications», 18 - 21 декабря, 2017 года, г. Канкун, Мексика.
\end{itemize}

{\contribution} Научные результаты, представленные в диссертационной работе, получены соискателем самостоятельно и при непосредственном участии.

%\publications\ Основные результаты по теме диссертации изложены в ХХ печатных изданиях~\cite{Sokolov,Gaidaenko,Lermontov,Management},
%Х из которых изданы в журналах, рекомендованных ВАК~\cite{Sokolov,Gaidaenko}, 
%ХХ --- в тезисах докладов~\cite{Lermontov,Management}.

\ifnumequal{\value{bibliosel}}{0}{% Встроенная реализация с загрузкой файла через движок bibtex8
    \publications\ Основные результаты по теме диссертации изложены в XX печатных изданиях, 
    X из которых изданы в журналах, рекомендованных ВАК, 
    X "--- в тезисах докладов.%
}{% Реализация пакетом biblatex через движок biber
%Сделана отдельная секция, чтобы не отображались в списке цитированных материалов
    \begin{refsection}%
        \printbibliography[heading=countauthornotvak, env=countauthornotvak, keyword=biblioauthornotvak, section=1]%
        \printbibliography[heading=countauthorvak, env=countauthorvak, keyword=biblioauthorvak, section=1]%
        \printbibliography[heading=countauthorconf, env=countauthorconf, keyword=biblioauthorconf, section=1]%
        \printbibliography[heading=countauthor, env=countauthor, keyword=biblioauthor, section=1]%
        \nocite{globecom,electosvyasen-2017, bulletin-rzd, itivs-2017, icmla-2017, ent, ent-2017}
        \nocite{ent, itivs-2017, electosvyasen-2017, ent-2017}
        \nocite{globecom, icmla-2017, bulletin-rzd}
        \publications\ Основные результаты по теме диссертации изложены в \arabic{citeauthor} печатных изданиях, 
        \arabic{citeauthorvak} из которых изданы в журналах, рекомендованных ВАК, 
        \arabic{citeauthorconf} "--- в тезисах докладов.
    \end{refsection}
}
% При использовании пакета \verb!biblatex! для автоматического подсчёта
% количества публикаций автора по теме диссертации, необходимо
% их здесь перечислить с использованием команды \verb!\nocite!.
    

