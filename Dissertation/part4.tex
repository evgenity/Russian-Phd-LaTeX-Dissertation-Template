\chapter{Разработка специального математического и алгоритмического обеспечения систем обработки информации в эргатических системах} \label{chapt4}

\begin{section}{Описание проектной команды}
\todo{Международный исследовательский университет, созданный при поддержке Массачусетского технологического института по новой для России модели, одной из основных целей которого является привлечение к работе специалистов высшего уровня, а также трансфер технологий из любой страны мира, что позволяет быстро наращивать компетенцию международного уровня в целевой области знаний.
Сложившийся крупный (около 100 ППС и научных сотрудников, 60 аспирантов) коллектив специалистов мирового уровня в области информационных технологий. За 2013 - 2017 гг исследователями Сколтеха опубликовано:
900+ публикаций по компьютерным наукам, информатике и смежным специальностям, проиндексированных в базе данных http://dblp.uni-trier.de/ ;}
\todo{60+ публикаций в ведущих профильных журналах (IF>3) и 80+ публикаций в ведущих профильных конференциях (h5>30).
Тесные связи и опыт совместных исследований (в том числе по международным грантам и проектам) с российскими и зарубежными научными и образовательными организациями (ВШЭ, МФТИ, РЭШ, Massachusetts Institute of Technology, New York University, TU Berlin, University of Maryland, Israel Institute of Technology др.).}
\todo{Успешный опыт работы с крупными коммерческими компаниями (Сбербанк, Airbus, Аэрофлот, Яндекс, Газпром нефть), в том числе специализирующихся в области беспроводных технологий и интернета вещей (Huawei, Philips) и др., инновационными технологическими компаниями и стартапами (Стриж Телематика, NWave, Цифра, Datadvance, Inspector Cloud, Vision Labs и др.).}
\todo{Культура активного междисциплинарного взаимодействия между подразделениями внутри Сколтеха (Центры Науки Инноваций и Образования (ЦНИО): вычислительных технологий, новых производственных технологий, биотехнологий, материаловедения, нефтегазовых разработок, энергетических систем).}
\todo{Глубокая интеграция в экосистему инновационного центра Сколково.
Новейшая материально-техническая база, включающая в себя все необходимое оборудование для проведения исследований и разработок, в том числе центр обработки данных, вновь созданная и полностью укомплектованная специализированная лаборатория Интернета вещей, а \todo{также опытные участки металлообработки и аддитивного производства, оснащенные системами промышленного IoT.}
Ожидается, что в состав консорциума, создаваемого на базе Сколтеха, войдет ряд ведущих российских ВУЗов и НИИ (разработчиков технологий беспроводной связи и интернета вещей и соответствующих образовательных программ) и коммерческих компаний-партнеров (заказчиков и потребителей технологий).}
\end{section}%{Описание проектной команды}


Обсуждение вопросов внедрения алгоритмов ML в промышленности
http://ieeexplore.ieee.org/stamp/stamp.jsp?arnumber=7293887\&tag=1

Методология применения машинного обучения в обрабатывающей промышленности (с картинками)
http://ieeexplore.ieee.org/document/8051033/
