\chapter*{Список сокращений и условных обозначений}             % Заголовок
\addcontentsline{toc}{chapter}{Список сокращений и условных обозначений}  % Добавляем его в оглавление
\noindent
\addtocounter{table}{-1}% Нужно откатить на единицу счетчик номеров таблиц, так как следующая таблица сделана для удобства представления информации по ГОСТ
%\begin{longtabu} to \dimexpr \textwidth-5\tabcolsep {r X}
\begin{longtabu} to \textwidth {r X}
% % Жирное начертание для математических символов может иметь
% % дополнительный смысл, поэтому они приводятся как в тексте
% % диссертации
% $\begin{rcases}
% a_n\\
% b_n
% \end{rcases}$  & 
% \begin{minipage}{\linewidth}
% коэффициенты разложения Ми в дальнем поле соответствующие
% электрическим и магнитным мультиполям
% \end{minipage}
% \\
% ${\boldsymbol{\hat{\mathrm e}}}$ & единичный вектор \\
% $E_0$ & амплитуда падающего поля\\
% $\begin{rcases}
% a_n\\
% b_n
% \end{rcases}$  & 
% коэффициенты разложения Ми в дальнем поле соответствующие
% электрическим и магнитным мультиполям ещё раз, но без окружения
% minipage нет вертикального выравнивания по центру.
% \\
% $j$ & тип функции Бесселя\\
% $k$ & волновой вектор падающей волны\\

% $\begin{rcases}
% a_n\\
% b_n
% \end{rcases}$  & 
% \begin{minipage}{\linewidth}
% \vspace{0.7em}
% и снова коэффициенты разложения Ми в дальнем поле соответствующие
% электрическим и магнитным мультиполям, теперь окружение minipage есть
% и добавленно много текста, так что описание группы условных
% обозначений значительно превысило высоту этой группы... Для отбивки
% пришлось добавить дополнительные отступы.
% \vspace{0.5em}
% \end{minipage}
% \\
% $L$ & общее число слоёв\\
% $l$ & номер слоя внутри стратифицированной сферы\\
% $\lambda$ & длина волны электромагнитного излучения
% в вакууме\\
% $n$ & порядок мультиполя\\
% $\begin{rcases}
% {\mathbf{N}}_{e1n}^{(j)}&{\mathbf{N}}_{o1n}^{(j)}\\
% {\mathbf{M}_{o1n}^{(j)}}&{\mathbf{M}_{e1n}^{(j)}}
% \end{rcases}$  & сферические векторные гармоники\\
% $\mu$  & магнитная проницаемость в вакууме\\
% $r,\theta,\phi$ & полярные координаты\\
% $\omega$ & частота падающей волны\\

\textbf {АРМ} & Автоматизированное рабочее место\\
\textbf {АПК-ДК} & Аппаратно-програмный комплекс диспетчерского контроля \\
\textbf {ЦП} & Управление пути Центральной Дирекции Инфрастуктуры (ЦДИ)\\
\textbf {ЦШ} & Управление Автоматики и телемеханики ЦДИ\\
\textbf {ЦУСИ} & Центр управления содержанием инфраструктуры Центральной дирекции инфраструктуры - филиала ОАО "РЖД" \\
\textbf {ЦКИ} & Департамент информатизации \\
\textbf {ПКБ И} & Проектно-конструкторское бюро инфраструктуры \\
\textbf {МЦК} & Московское центральное кольцо \\
\textbf {СЦБ} & Cигнализация, централизация, блокировка \\
\textbf {ЖАТ}   & Устройства железнодорожной автоматики и телемеханики \\
\textbf {ШНС} & Cтарший электромеханик СЦБ или связи \\
\textbf {ШЧД} & Диспетчер дистанции или дежурный инженер дистанции \\
\textbf {Ш} & Cлужба сигнализации и связи \\
\textbf {ШЧ} & Дистанция сигнализации, централизации и блокировки \\
\textbf {ПЧ} & Дистанция пути, начальник дистанции пути \\
\textbf {ЭЧ} & Дистанция электроснабжения, начальник дистанции электроснабжения \\
\textbf {ТЧ} & Тяговая часть (локомотивное депо); начальник депо \\
\textbf{AMC}	&	Adaptive Modulation and Coding \\
\textbf{CQI}	&	Channel Quality Indicator \\
\textbf{DCE}	&	Direct Code Execution \\
\textbf{DCI}	&	Downlink Control Information \\
\textbf{DRX}	&	Discontinuous transmission/reception \\
\textbf{ENB}	&	Evolved Node B, базовая станция LTE \\
\textbf{EPC}	&	Evolved Packet Core, сеть оператора LTE \\
\textbf{E-UTRAN}	&	Evolved Universal Terrestrial Radio Access Network,  сеть радиодоступа LTE \\
\textbf{FDPS}	&	Frequency Domain Packet Scheduling \\
\textbf{GBR}	&	Guaranteed Bit Rate \\
\textbf{HARQ}	&	Hybrid Automatic Retry reQuest \\
\textbf{hENB}	&	Home ENB, малая базовая станция LTE \\
\textbf{IR}	&	Incremental Redundancy \\
\textbf{LTE}	&	Long Term Evolution \\
\textbf{MAC}	&	Medium Access Control \\
\textbf{MCS}	&	Modulation and Coding Scheme \\
\textbf{MME}	&	Mobility Management Entity \\
\textbf{OFDM}	&	Orthogonal Frequency Division Multiplexing \\
\textbf{PCFICH}	&	Physical Control Format Indicator Channel \\
\textbf{PCI}	&	Physical Cell Identity \\
\textbf{PDCCH}	&	Physical Downlink Control Channel \\
\textbf{PDCP}	&	Packet Data Convergence Protocol \\
\textbf{PDN}	&	Packet Data Network \\
\textbf{PDSCH}	&	Physical Downlink Shared Channel \\
\textbf{PGW}	&	Packet Gateway \\
\textbf{PoP}	&	Point of Presence, точка подключения к сети оператора \\
\textbf{PSS}	&	Primary Synchronization Signal \\
\textbf{QoS}	&	Quality of Service \\
\textbf{RB}	&	Resource Block \\
\textbf{RLC}	&	Radio Link Control \\
\textbf{RRC}	&	Radio Resource Control \\
\textbf{RRM}	&	Radio Resource Management, управление ресурсами радиоканала \\
\textbf{RS}	&	Reference Signal \\
\textbf{RSRP}	&	Reference Signal Received Power \\
\textbf{RSRQ}	&	Reference Signal Received Quality \\
\textbf{RSSI}	&	Received Power Strength Indicator \\
\textbf{SaaS}	&	Software as a Service \\
\textbf{SC-FDMA}	&	Single Carrier Frequency Division Multiplexing \\
\textbf{SGW}	&	Service Gateway \\
\textbf{SINR}	&	Signal to Interference plus Noise Ratio \\
\textbf{SRS}	&	Sounding Reference Signal \\
\textbf{TCP}	&	Transmission Control Protocol \\
\textbf{TTI}	&	Transmission Time Interval \\
\textbf{UDP}	&	User Datagram Protocol \\
\textbf{UE}	&	User Equipment, абонентское устройство \\


\end{longtabu}
